\documentclass{article}
\usepackage[utf8]{inputenc} % Obsługa polskich znaków w UTF-8
\usepackage[T1]{fontenc}    % Ustawienie kodowania czcionki
\usepackage[polish]{babel}  % Język polski
\usepackage{amsmath}
\usepackage{graphicx}
\usepackage{float} % Pakiet do wymuszania pozycji


\title{Sprawozdanie z Zadania Numerycznego 5}
\author{Paweł Nykiel}
\date{Listopad 2024}

\begin{document}

\maketitle

\section{Wprowadzenie}
Cel zadania to przybliżone rozwiązanie układu równań liniowych używając dwóch metod iteracyjnych – Jacobiego oraz Gaussa-Seidela.
Dla zadanej macierzy o rozmiarze \( N = 200 \) analizujemy dokładność i zbieżność obu metod dla różnych wartości \( d \) na diagonali
i losowych wektorów startowych. Na wykresie rysujemy różnicę poniędzy dokładnym rozwiązaniem i przybliżeniem w kolejnych iteracjach
dla obu metod.




\section{Struktura Macierzy}
Macierz \( A \) jest zdefiniowana jako:
\[
A = \begin{pmatrix}
d & 0.5 & 0.1 & & &  \\
0.5 & d & 0.5 & 0.1 & & \\
0.1 & 0.5 & d & 0.5 & 0.1 & \\
& \ddots & \ddots & \ddots & \ddots & \\
& & 0.1 & 0.5 & d & 0.5 \\
& & & 0.1 & 0.5 & d \\
\end{pmatrix}
\]
gdzie \( d \) jest elementem diagonalnym. Wektor prawej strony \( b \) jest określony przez:
\[
b = \begin{pmatrix} 1 \\ 2 \\ 3 \\ \vdots \\ N \end{pmatrix}
\]
dla \( N = 200 \).

\section{Metody}

\subsection{Metoda Jacobiego}
Metoda Jacobiego to iteracyjny algorytm, w którym każdy element wektora rozwiązania jest aktualizowany
niezależnie na podstawie wartości z poprzedniej iteracji. Wzór iteracyjny zadany jest przez:
\[
x_i^{(k+1)} = \frac{1}{a_{ii}} \left(b_i - \sum_{j \neq i} a_{ij} x_j^{(k)} \right)
\]
gdzie \( x_i^{(k+1)} \) to zaktualizowana wartość \( x_i \) w kolejnej iteracji.

\subsection{Metoda Gaussa-Seidla}
Metoda Gaussa-Seidla jest bardzo podobna do metody Jacobiego, jednakże aktualizacje każdego elementu zachodzą
sekwencyjnie w ramach tej samej iteracji, natychmiast korzystając z nowo obliczonych wartości. Wzór iteracyjny zadany jest przez:
\[
x_i^{(k+1)} = \frac{1}{a_{ii}} \left(b_i - \sum_{j < i} a_{ij} x_j^{(k+1)} - \sum_{j > i} a_{ij} x_j^{(k)} \right)
\]

\section{Implementacja}
Zaimplementowano skrypt w Pythonie (`num5.py`), który wykorzystuje biblioteki `numpy` do operacji algebraicznych oraz `matplotlib` do tworzenia wykresów.
Macierz \( A \) została zbudowana w taki sposób, aby generować tylko niezerowe elementy, optymalizując zużycie pamięci dla dużych wartości \( N \).

\subsection{Kryterium Zbieżności}
Dla obu metod zakładamy zbieżność, gdy różnica norm między wektorem rozwiązania kolejnych iteracji spada poniżej progu \(10^{-13}\).

\section{Wyniki}

% Wymuszenie umieszczenia wszystkich wykresów na początku sekcji
\begin{figure}[H]
    \centering
    \includegraphics[width=1.2\textwidth]{myplot} % Główny wykres
    \label{fig:main-plot}
\end{figure}

\begin{figure}[H]
    \centering
    \begin{minipage}{0.7\textwidth}
        \centering
        \includegraphics[width=\textwidth]{myplot121} % Pomniejszony wykres 1
        \label{fig:plot1}
    \end{minipage}
    \hfill
    \begin{minipage}{0.7\textwidth}
        \centering
        \includegraphics[width=\textwidth]{myplot124} % Pomniejszony wykres 2
        \label{fig:plot2}
    \end{minipage}
    \begin{minipage}{0.7\textwidth}
        \centering
        \includegraphics[width=\textwidth]{myplot13} % Pomniejszony wykres 1
        \label{fig:plot3}
    \end{minipage}
    \hfill
\end{figure}


\begin{itemize}

    \item \textbf{Warunki początkowe}: Dla zadanych wartości elementów diagonalnych \( d \) metoda była uruchamiana z losowym wektorem startowym,
    niezależnie generowanym dla każdego przypadku testowego. W ten sposób zbadaliśmy wpływ wartości początkowych na zbieżność metod iteracyjnych.
    \item \textbf{Graficzna Reprezentacja}:
    Wykresy przedstawiają normę różnicy między bieżącym przybliżeniem a dokładnym rozwiązaniem dla każdej iteracji oraz dla różnych wartości elementu \( d \) na diagonali.
    Umożliwia to wizualne porównanie tempa zbieżności metod Jacobiego i Gaussa-Seidla.
    \item \item \textbf{Zbieżność i Porównanie}:
    \begin{itemize}
        \item \textbf{Metoda Jacobiego}: Zazwyczaj wymaga więcej iteracji niż metoda Gaussa-Seidla, aby osiągnąć zbieżność.
         Wizualizacje pokazują stopniowe, liniowe zmniejszanie się błędu wraz z kolejnymi iteracjami.
        \item \textbf{Metoda Gaussa-Seidla}: Zbiega szybciej dzięki natychmiastowej aktualizacji elementów wektora,
         często wymagaja mniej iteracji niż metoda Jacobiego.
    \end{itemize}
    \item \textbf{Wpływ Parametru \( d \)}:
    Wartość \( d \) wpływa na tempo zbieżności. Gdy \( d \) jest zbyt małe, a w szczególności gdy zbliża się do 1.2,
    metoda Jacobiego napotyka trudności ze zbieżnością. W ogólności jednak, obie metody zbiegają szybciej dla większych wartości \( d \).
\end{itemize}

\section{Dyskusja}
Metoda Gaussa-Seidla zazwyczaj zbiega szybciej niż metoda Jacobiego, co nie znaczy, że metoda Jacobiego jest nieskuteczna.
Dla określonych struktur macierzy i punktów startowych metoda Jacobiego może być dobrą alternatywą.
Dla obu metod zbieżność jest gwarantowana przy odpowiednio dobranym parametrze \( d \). Musi on spełniać warunek stworzenia macierzy przekątniowo dominującej.
Widzimy, że dla wartości \( d \) bliskich 1.2 metoda Jacobiego zaczyna napotykać trudności ze zbieżnością, co jest zgodne z oczekiwaniami.


\section{Wnioski}
Dokument przedstawia metody iteracyjne oraz ich skuteczność w rozwiązywaniu układów równań.
Metoda Gaussa-Seidla zazwyczaj osiąga zbieżność przy mniejszej liczbie iteracji niż metoda Jacobiego.
Zatem w sytuacjach, gdy zależy nam na szybkości zbieżności oraz efektywności obliczeniowej, metoda Gaussa-Seidla jest lepszym wyborem.
Jednakże, dla podanej struktury macierzy obie metody zbiegają, pod warunkiem, że macierz jest przekątniowo dominująca.

\end{document}
